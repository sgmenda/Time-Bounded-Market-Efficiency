\documentclass[11pt]{article}

% margins
\usepackage[margin=1in]{geometry}

% fonts
\usepackage{fourier}
\usepackage[T1]{fontenc}
\usepackage{inconsolata}
\usepackage{amssymb}

% color
\usepackage{xcolor}

% hyperlinks
\usepackage[colorlinks=true, allcolors=blue]{hyperref}

% math
\usepackage{amsmath, mathtools, amsthm}
\newtheorem{theorem}{Theorem}[section]
\newtheorem{lemma}[theorem]{Lemma}
\theoremstyle{definition}
\newtheorem{definition}[theorem]{Definition}

\begin{document}

\title{\bfseries Time-Bounded Market Efficiency}
\author{%
    Sanketh\thanks{%
        \url{https://c1own.com}%
    }
}
\date{DRAFT:\;\today}
\maketitle

\begin{abstract}
    We study market efficiency relative to time-bounded agents.  Following
    Hasanhodzic et al.~\cite{HasanhodzicLV11}, we define a market to be
    \emph{$T$-efficient} if no strategy using at most $T$ time can have expected
    return in excess of equilibrium return. We show that, for polynomial $T$,
    $T$-efficient markets can have market spikes assuming that there exist
    secure pseudorandom functions. Further, we show that a strategy $A$ with
    access with asymptotically more time than strategy $B$ can outperform
    strategy $B$. Our results can be seen as evidence that volatile markets do
    not rule out market efficiency.
\end{abstract}

\section{Introduction}

Informally speaking, a market is \emph{efficient} if it reflects all available
information. One way to formalize this is using the \emph{random walk model}
which posits that successive price changes are independent and identically
distributed; i.e., prices follow a random walk. See Fama~\cite{Fama70} for a
survey of theoretical and empirical work on the efficient market hypothesis.

In this paper, we are going to use the computational definition of market
efficiency due Hasanhodzic et al.~\cite{HasanhodzicLV11}.

\begin{definition}[\cite{HasanhodzicLV11}]
    A market is \emph{efficient} with respect to resources $S$ if no strategy
    using resources $S$ can generate a substantial profit.
\end{definition}

But, in this paper, we are going to fix the resources under consideration to be time.

\begin{definition}
    A market is \emph{$T$-efficient} if no strategy using at most $T$ time can
    have expected return in excess of equilibrium return. 
\end{definition}

\bibliographystyle{alphaurl}
\bibliography{market.bib}

\newpage
\appendix
\section{My Email}

This is an email I wrote one of the authors of \cite{HasanhodzicLV11}. This
paper is an expansion of this thought.

\begin{quote}
    Perhaps this is obvious from your paper, but one can shove hard problems
    into predicting the sign of next return. The native “let’s put 3SAT into it”
    doesn't seem to work but one can instead use a randomly seeded PRF and let
    the return of the ith day be the output of the PRF on input i.  More
    precisely, you pick a subexponentially secure PRF and run it only
    polynomially many times (assume that the stock dies after polynomially many
    days.) You also assume, as usual in TCS, that the agent is polynomial-time
    bounded.  Your main results also seem to hold in this model:
    \begin{enumerate}
        \item spikes are possible, and 
        \item an agent with exponential-time can break the PRF and make a lot of
            money. (Also, by messing with the PRF you can get an analog of your
            Claim 2.5 and a more fine-grained version of (2).) (The
            “feeding-off” stuff may also be possible in this model, but I
            haven’t thought too much about it.) 
    \end{enumerate}
    (Technical aside: as usual, the PRF is public knowledge but the seed is kept
    private.) Of course, all these results would be conditioned on complexity
    theoretic or cryptographic assumptions. 
\end{quote}

\end{document}
